\documentclass[12pt]{article}
\pagestyle{empty}
\newcommand\tab[1][1cm]{\hspace*{#1}}
\usepackage[margin=1in]{geometry}
\usepackage{array}
\usepackage{amsmath}
%\usepackage{cyrillic}
\usepackage{graphicx}
\usepackage{subcaption}
\usepackage{float}
\usepackage{bm}
\usepackage{tikz}

\thispagestyle{empty}

\begin{document}
\begin{center}
\large\bf Networked Life: Homework 6
\medskip\\
Guo Ziqi - 1000905\\
Zhao Juan - 1000918\\
Zhang Hao - 1000899
\end{center}

\begin{enumerate}

\item{} \textbf{Answer:}
\medskip\\
Based on how the graph is connected, we can tell that node 6 is the destination. Therefore, the steps of the Bellman-Ford Algorithm are tabulated below:
\begin{table}[ht]

\centering
\begin{tabular}{c c c c c c c}
\hline
t & d6 & d5 & d4 & d3 & d2 & d1\\ \hline
t=0 & 0 & $\infty$ & $\infty$ & $\infty$ & $\infty$ & $\infty$ \\
t=1 & 0 & 5 & 2 & 9 & $\infty$ & 8 \\
t=2 & 0 & 5 & 2 & 9 & 3 & 7 \\
t=3 & 0 & 5 & 2 & 9 & 3 & 7 \\
\hline
\end{tabular}
\caption{Steps of the Bellman-Ford Algorithm}
\label{table: Bellman-Ford}
\end{table}\\
From $t=2$ to $t=3$, the shortest distances were not updated any more. Therefore we obtained the shortest path from all points to node 6.
\bigskip\\
\item{} \textbf{Answer:}
\begin{itemize}
\item[(1)]$N_d = C/1 = C$.
\item[(2)]$P(X>N_d)<\gamma$ is equivalent with $P(X\leq N_d)\geq1-\gamma$. So we just have to choose $N_s$ such that:
$$P(X\leq N_d)=\sum_{i=1}^{N_d}\binom{N_s}{i}p^i(1-p)^{N_s-i}\geq1-\gamma$$
Plugging in $N_d=C$ and $\gamma=0.01$ and $p=0.1$, we increment $N_s$ from $C$ to find the last number that still maintains $P(X\leq C)\geq0.99$. This number is the maximum number of users in the system. Following this method, we obtained: $N_s = 50$ when $C = 10$; $N_s = 122$ when $C = 20$; $N_s = 200$ when $C = 30$. There appears to be a linear relationship between $N_s$ and the value of $C$. $\sum_{i=N_d+1}^{N_s}P(X=i)$ when $N_d=10$ is 0.00935.
\item[(3)]Let's take the case when $C=10$ as an example. $N_s=50$ in this case. 
\medskip\\
In System A, an $N_s$ equal to 122 can be achieved. In System B, the total $N_s$ is simply 2 times the original $N_s=50$, which is $100$. Comparing these two values, System A can fit the largest amount of users. This is because when the resources are pooled together, the system is more flexible and tolerant. In System B, there might be a case where one link is crowded but the one link is empty. When resource pooling is used, such uneven distribution of demand can be eliminated, resulting in a more robust system.
\end{itemize}
\item{} \textbf{Answer:}
\medskip\\
We can re-express $1/E[n,\rho]$ using formula (1):
$$\dfrac{1}{E[n,\rho]}=\dfrac{e^\rho\sum_\rho^\infty\dfrac{x^ne^{-x}}{n!}dx}{\rho^n/n!}$$
Cancelling $n!$ and plugging in $t$, we get:
$$\dfrac{1}{E[tn,t\rho]}=\dfrac{e^{t\rho}\sum_{t\rho}^\infty x^{tn}e^{-x}dx}{t\rho^{tn}}=\dfrac{\sum_{t\rho}^\infty x^{tn}e^{t\rho-x}dx}{t\rho^{tn}}$$
Re-indexing the range of the summation, we get:
$$\dfrac{1}{E[tn,t\rho]}=\dfrac{\sum_0^\infty(x+t\rho)^{tn}e{t\rho-x-t\rho}dx}{t\rho^{tn}}=\sum_0^\infty(\dfrac{x+t\rho}{t\rho})^{tn}e^{-x}$$
Only the middle term $(\dfrac{x+t\rho}{t\rho})^{tn}$ contains $t$. So we only need to compare which term, $(x+t\rho)^{tn}$ or $(t\rho)^{tn}$ grows faster as $t$ gets large. It is obvious that given the same index, the nominator will increase faster than the denominator does. Therefore, $1/E[tn,t\rho]$ is increasing with $t$. As a result, it has been proven that the original function $E[tn,t\rho]$ is strictly decreasing in $t$.


\end{enumerate}
\end{document}


 
