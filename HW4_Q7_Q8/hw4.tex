\documentclass[12pt]{article}%
\usepackage{amsfonts}
\usepackage{fancyhdr}
\usepackage[a4paper, top=2.5cm, bottom=2.5cm, left=2.2cm, right=2.2cm]%
{geometry}
\usepackage{times}
\usepackage{amsmath}
\usepackage{amssymb}
\usepackage{graphicx}%
\setcounter{MaxMatrixCols}{30}
\newtheorem{theorem}{Theorem}
\newtheorem{acknowledgement}[theorem]{Acknowledgement}
\newtheorem{algorithm}[theorem]{Algorithm}
\newtheorem{axiom}{Axiom}
\newtheorem{case}[theorem]{Case}
\newtheorem{claim}[theorem]{Claim}
\newtheorem{conclusion}[theorem]{Conclusion}
\newtheorem{condition}[theorem]{Condition}
\newtheorem{conjecture}[theorem]{Conjecture}
\newtheorem{corollary}[theorem]{Corollary}
\newtheorem{criterion}[theorem]{Criterion}
\newtheorem{definition}[theorem]{Definition}
\newtheorem{example}[theorem]{Example}
\newtheorem{exercise}[theorem]{Exercise}
\newtheorem{lemma}[theorem]{Lemma}
\newtheorem{notation}[theorem]{Notation}
\newtheorem{problem}[theorem]{Problem}
\newtheorem{proposition}[theorem]{Proposition}
\newtheorem{remark}[theorem]{Remark}
\newtheorem{solution}[theorem]{Solution}
\newtheorem{summary}[theorem]{Summary}
\newenvironment{proof}[1][Proof]{\textbf{#1.} }{\ \rule{0.5em}{0.5em}}

\newcommand{\Q}{\mathbb{Q}}
\newcommand{\R}{\mathbb{R}}
\newcommand{\C}{\mathbb{C}}
\newcommand{\Z}{\mathbb{Z}}

\begin{document}

\title{Homework 4}
\author{Guo Ziqi - 1000905\\Zhao Juan -1000918 \\Zhang Hao -1000899 }
\date{\today}
\maketitle
\section{Exercise 1 (Exercise 7.3 in textbook)}
\subsection{Solution(a)}
$$A = \left(\begin{matrix} 
0&0&1&1&0&0&0&0
\\ 0&0&0&0&1&1&0&0
\\0&0&0&0&0&0&1&0
\\0&0&0&0&0&0&1&1
\\0&0&0&0&0&0&1&1
\\0&0&0&0&0&0&0&1
\\0&0&0&0&0&0&0&0
\\0&0&0&0&0&0&0&0
 \end{matrix}\right)$$
\subsection{Solution(b)}
$$C = \left(\begin{matrix} 
0&0&0&0&0&0&0&0
\\0&0&0&0&0&0&0&0
\\0&0&1&1&0&0&0&0
\\0&0&1&1&0&0&0&0
\\0&0&0&0&1&1&0&0
\\0&0&0&0&1&1&0&0
\\0&0&0&0&0&0&3&2
\\0&0&0&0&0&0&2&3
 \end{matrix}\right)$$
 $$C_{78}=2, C_{75}=0$$ The physical interpretation is the number of common nodes pointing to both i and j. For node 7 and node 8, there are 2 common nodes pointing to them. For node 7 and node 5, there is zero common node.
\subsection{Solution (c)}
$$A^2 = \left(\begin{matrix} 
0&0&0&0&0&0&2&1
\\0&0&0&0&0&0&1&2
\\0&0&0&0&0&0&0&0
\\0&0&0&0&0&0&0&0
\\0&0&0&0&0&0&0&0
\\0&0&0&0&0&0&0&0
\\0&0&0&0&0&0&0&0
\\0&0&0&0&0&0&0&0
 \end{matrix}\right),
 A^3 = \left(\begin{matrix} 
0&0&0&0&0&0&0&0
\\0&0&0&0&0&0&0&0
\\0&0&0&0&0&0&0&0
\\0&0&0&0&0&0&0&0
\\0&0&0&0&0&0&0&0
\\0&0&0&0&0&0&0&0
\\0&0&0&0&0&0&0&0
\\0&0&0&0&0&0&0&0
 \end{matrix}\right)$$
 \\
 For $A^3$, every entry is 0. For $A^m$, where m = 1,2,..., each entry $A^m_{ij}$ means the number of shorted path with length m from node i to node j. For example, from node 1 to node 7, there are 2 shortest paths with length 2. Hence $A^2_{17}=2$. However, since for each pair of node the shortest path is smaller or equal to 2, the entry for $A^3$ is zero.
 
\section{Exercise 2 (Exercise 8.1 in textbook)}
\subsection {Solution(a)}
$$Degree =\left(\begin{matrix}3&2&3&3&3\end{matrix}\right)$$
$$Closeness =\left(\begin{matrix}0.8&0.2&0.8&0.8&0.8\end{matrix}\right)$$
$$Eigenvector=\left(\begin{matrix}0.45579856 &0.31921209 &0.49122245&0.49122245 &0.45579856\end{matrix}\right)$$
To compute the eignenvector centrality, firstly we wrote down the adjacency matrix A:\\
$$A = \left(\begin{matrix} 0&1&1&1&0
\\1& 0&0&0&1
\\1&0&0&1&1
\\1&0&1&0&1
\\0&1&1&1&0\end{matrix}\right)$$
 After that we calculated A's absolute eigenvalues:\\
$$Eigenvalues= \left(\begin{matrix}2.856e+00&2.177e+00&7.396e-17&3.216e-01&1.000e+00\end{matrix}\right)$$ Since 2.856e+00 is the largest eigenvalue, its corresponding eigenvector $$\left(\begin{matrix}0.45579856 &0.31921209 &0.49122245&0.49122245 &0.45579856\end{matrix}\right)$$ is the eigenvector centrality.
\subsection {Solution(b)}
Compute the node betweenness between node 2 and node3.\\
Node 2: 1+1/3+3=13/3\\
Node 3: 1+1/3+1+2=13/3
\subsection {Solution(c)}
Compute the link betweenness(3,4) and (2,5).\\
(3,4)=1\\
(2,5)=1/3+1/2+1/2+1=7/3
\section{Exercise 3 (Exercise 8.2 in textbook) }
\subsection{Solution(a)}
Run the contagion model with node 1 initialized at state-1 and the other nodes initialized to state-0. The result is $[0  1 -1 -1 -1 -1  1  2]$. 
\medskip\\Here -1 means healthy, 0 means node 1 is infected initially, 1 means both node 2 and 7 are infected after the first round, 2 means node 8 is infected in the second round.
\subsection{Solution(b)}
Run the contagion model with node 3 initialized at state-1 and the other nodes initialized to state-0. The result is $[1 2 0 3 3 4 2 3]$.
\medskip\\ Here 0 means node 3 is infected initially, 1 means both node 1 is infected after the first round, 2 means both node 2 and node 7 are infected after the second round.3 means both node 4,5,8 are infected after the third round. 4 means node 6 is infected after the fourth round.
\subsection{Solution(c)} 
For section(b), when node 1 is initialized to state-1, node 3,4,5,6 forms a cluster of density 0.75, which is higher than p=0.3. This prevents complete flipping. However, in section(c), when node 3 is initialized to state-1, there is no cluster of density higher than 0.7. Therefore, complete flipping happened for this case.
\section{Exercise 4 (Exercise 8.3 in textbook)}     
After 200 iterations, S(t), I(t), R(t) converge to $\left(\begin{matrix}0.3334&0.0364&0.6301\end{matrix}\right)$. This means eventually, majority of the people will stay immune to the disease. This is because although state-R can transit to state-S, the rate of transition is only 1/50, much lower than the rate infected people get immune. Thus, only a relatively small fraction of total population will be susceptible to the disease in the long run.

\section{Exercise 5 (Exercise 8.4 in textbook)}
Firstly, we contructed our adjacency matrix W with weights:
$$W = \left(\begin{matrix} 0&0.7&0.5&0.9&0
\\0.7&0&0&0&0.8
\\0.5&0&0&0.5&0.2
\\0.9&0&0.5&0&0.3
\\0&0.8&0.2&0.3&0\end{matrix}\right)$$
Then our matrix A is:
$$A = \left(\begin{matrix} 3.1&0.3&0.5&0.1&1
\\0.3&2.5&1&1&0.2
\\0.5&1&2.2&0.5&0.8
\\0.1&1&0.5&2.7&0.7
\\1&0.2&0.8&0.7&2.3\end{matrix}\right)$$
We went on with calculating matrix C:
$$C = \left(\begin{matrix} 0.38&-0.05&-0.01&0.05&-0.17
\\-0.05&0.57&-0.25&-0.19&0.12
\\-0.01&-0.25&0.64&0.03&-0.20
\\0.05&-0.19&0.03&0.48&-0.16
\\-0.17&0.12&-0.20&-0.16&0.62\end{matrix}\right)$$
Finally, after calculating T,R and N, we derived the information centrality for all the nodes:
$$C_I = \left(\begin{matrix} 1.19&0.97&0.91&1.07&0.93\end{matrix}\right)$$
If we consider the weights of edges to be distances between nodes, then matrix C basically captures the distances of the shortest paths in the graph. The smaller the distance from one node to the other nodes, the more important and central this particular node is. The most central node has the smallest sum of weights on paths to other nodes on average.
\end{document}